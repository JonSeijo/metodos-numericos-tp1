\section{Conclusión}

En primer lugar vimos cómo calibrar el sistema para adaptarlo a cualquier ángulo de luz, obteniendo resultados similares a los provistos por la cátedra. \\

Resolvimos sistemas lineales para encontrar los vectores normales a la superfice. Lo hicimos utilizado el algoritmo de eliminación gaussiana. Luego nos aprovechamos de la estructura de nuestras ecuaciones para mostrar que existe la factorización LU y la utilizamos en nuestro problema. Comprobamos en la experimentación que LU es bastante más eficiente que Gauss, sin embargo, el tiempo que se tarda en el cálculo de normales se encuentra en el orden de los milisegundos para ambos métodos, mientras que el tiempo total se encuentra en el orden de los minutos, por lo que para nuestro problema es indistinto cuál usar. \\

Planteamos cómo sería un posible sistema matricial para el cálculo de profundidades a partir de las ecuaciones de los planos tangentes y vimos que cumplía ciertas propiedades. Para resolver el nuevo sistema, propusimos una nueva estructura de datos y resolvimos el sistema utilizando el algoritmo de Cholesky. \\

Finalmente, mostramos los resultados finales obtenidos y realizamos experimentaciones para entender mejor cómo se comporta nuestro sistema. Consideramos las diferencias que encontramos y concluimos que si bien no fue perfecto, aplicamos de manera satisfactoria la técnica de fotometría estéreo. \\