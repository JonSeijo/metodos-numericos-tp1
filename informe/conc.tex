\section{Conclusión}

En primer lugar vimos cómo calibrar el sistema para adaptarlo a cualquier ángulo de luz, obteniendo resultados similares a los provistos por la cátedra. \\

Resolvimos sistemas lineales para encontrar los vectores normales a la superfice, utilizado el algoritmo de eliminación gaussiana. Luego nos aprovechamos de la estructura de nuestras ecuaciones para mostrar que existe la factorización LU y la utilizamos en nuestro problema. Comprobamos en la experimentación que LU es bastante más eficiente que Gauss, sin embargo, el tiempo que se tarda en el cálculo de normales se encuentra en el orden de los milisegundos para ambos métodos, mientras que el tiempo total se encuentra en el orden de los minutos, por lo que para nuestro problema es indistinto cuál usar. \\

Planteamos cómo sería un posible sistema matricial para el cálculo de profundidades a partir de las ecuaciones de los planos tangentes y vimos que cumplía ciertas propiedades, en particular vimos que era simétrica y definida positiva. Al toparnos con la falta de memoria disponible, propusimos una nueva estructura de datos y resolvimos el sistema utilizando el algoritmo de Cholesky. \\

Mostramos los resultados finales obtenidos y realizamos experimentaciones para entender mejor cómo se comporta nuestro sistema. Realizamos mediciones en la calibración de luces para ver qué tan acertadas estaban, y vimos cómo se modifica el cálculo de normales con una mala calibración. Mostramos los resultados de diferentes modelos con distintas combinaciones de luces y comprobamos cuán importante es la elección. \\

Finalmente, realizamos mediciones de tiempos para ver cómo se comparan en eficiencia computacional los distintos algoritmos utilizados. Concluimos que si bien nuestros modelos finales no fueron perfectos, aplicamos de manera satisfactoria la técnica de fotometría estéreo. \\