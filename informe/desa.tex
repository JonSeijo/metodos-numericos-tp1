% Cita textual seccion Desarrolo:

% Deben explicarse los metodos numericos que utilizaron
% y su aplicacion al problema concreto involucrado en el trabajo practico.
% Se deben mencionar los pasos que siguieron para implementar
% los algoritmos, las dificultades que fueron encontrando y la
% descripcion de como las fueron resolviendo.

% Explicar tambien como fueron planteadas y realizadas las
% mediciones experimentales. Los ensayos fallidos, hipotesis y
% conjeturas equivocadas, experimentos y metodos malogrados deben
% figurar en esta seccion,
% con una breve explicacion de los motivos de estas fallas
% (en caso de ser conocidas).

\section{Cálculo de normales}

\todo[inline]{Explicar un poco mas que representa el sistema y como resolviendolo logramos lo de las normales. Ver que quiza esto convenga ponerlo luego de describir los metodos}

\subsection{Eliminación Gaussiana}

Tenemos un sistema listo para resolver, dónde sólo los $m_i$ son incógnitas:

% Sistema de matrices original
\[
\begin{pmatrix}
    s_{x}^{1} & s_{y}^{1} & s_{z}^{1} \\
    s_{x}^{2} & s_{y}^{2} & s_{z}^{2} \\
    s_{x}^{3} & s_{y}^{3} & s_{z}^{3}
\end{pmatrix}
\begin{pmatrix}
    m_{x} \\
    m_{y} \\
    m_{z}
\end{pmatrix}
=
\begin{pmatrix}
    I_{1} \\
    I_{2} \\
    I_{3}
\end{pmatrix}
\]

La pregunta ahora es ¿cómo lo resolvemos?. Dado que en principio no sabemos cómo, nos gustaría llevarlo a una forma equivalente que sea mas fácil de resolver. Podemos hacer esta conversión a un sistema equivalente usando el algortimo de eliminación de Gauss. Lo que hace este algortimo es llevar una matriz a su forma triangular superior, de dónde luego es muy sencillo hacer los despejes finales. \\

El pseudocódigo del algoritmo de Gauss es el siguiente:

\begin{algorithm}[H]
\begin{algorithmic}
\Function{EliminacionGaussiana}{Matriz M[$n$][$m$]}

    \For{$k \in [1..min(n,m)]$ }
        \For{$i \in [k+1..m]$ }

            \If{M[$k$][$k$] $\neq 0$}
                \State $mult \gets $M[$i$][$k$] $/$ M[$k$][$k$]
                \For{$j \in [k+1..n]$}

                    \State M[$i$][$j$] $\gets$ M[$i$][$j$] - mult*M[$k$][$j$]

                \EndFor
            \Else
                \State Hay un cero en la diagonal!
            \EndIf
        \EndFor
    \EndFor

\EndFunction
\end{algorithmic}
\end{algorithm}

Como puede verse, funciona correctamente solo \textbf{suponiendo que no hay ceros en la diagonal}. Es claro que puede modificarse para que realice intercambios de filas y no tenga el problema del cero, veamos que para nuestro problema no es importante. En nuestra implementación aplicaremos el algoritmo de Gauss en la siguiente matriz ampliada:

% Matriz ampliada
\[
\begin{pmatrix}[ccc|c]
    s_{x}^{1} & s_{y}^{1} & s_{z}^{1} & I_{1} \\
    s_{x}^{2} & s_{y}^{2} & s_{z}^{2} & I_{2} \\
    s_{x}^{3} & s_{y}^{3} & s_{z}^{3} & I_{3}
\end{pmatrix}
\]

Nuestros $s_{j}^{i}$ incialmente son todos distintos de cero. Además, tomamos todas las combinaciones posibles de tres luces y vimos que siempre son linealmente independientes, por lo tanto nunca vamos a hallar un cero en la diagonal y no es necesario aplicar permutaciones. Esto cobrará importancia cuando querramos aplicar factorización LU. \\

Dado que pudimos triangular correctamente la matriz ampliada, entonces ya estamos en condiciones de despejar de nuestra matriz de 3 x 4:

\[
\begin{pmatrix}[ccc|c]
    m_{1,1}   & m_{1,2} & m_{1,3} & m_{1, 4} \\
    0         & m_{2,2} & m_{2,3} & m_{2, 4} \\
    0         & 0       & m_{3,3} & m_{3, 4}
\end{pmatrix}
\]

\begin{algorithm}[H]
\begin{algorithmic}
\Function{Despejar}{Matriz M[$n$][$m$]}

    // En X se guardan los m-1 coeficientes solución (Recordemos que M es ampliada)
    \State X[$m-1$] $\gets \{\}$ \\

    \For{$j \in [1..m-1]$ }  (j es indice de columna)

        \If{M[$j$][$j$] $\neq 0$}

            \State X[$j$] $\gets$ M[$j$][$m$] / M[$j$][$j$]

            \For{$i \in [j-1 .. 0]$ }  (i es indice de fila)

                \State M[$i$][$m$] $\gets$ M[$i$][$m$] - ( M[$i$][$j$] * X[$j$] )

            \EndFor

        \Else
            \State Hay un cero en la diagonal!
        \EndIf
    \EndFor

    \State Retornar X

\EndFunction
\end{algorithmic}
\end{algorithm}


\subsection{Factorización LU}

Recordemos nuestro sistema para hallar las normales:
\[
\begin{pmatrix}
    s_{x}^{1} & s_{y}^{1} & s_{z}^{1} \\
    s_{x}^{2} & s_{y}^{2} & s_{z}^{2} \\
    s_{x}^{3} & s_{y}^{3} & s_{z}^{3}
\end{pmatrix}
\begin{pmatrix}
    m_{x} \\
    m_{y} \\
    m_{z}
\end{pmatrix}
=
\begin{pmatrix}
    I_{1} \\
    I_{2} \\
    I_{3}
\end{pmatrix}
\]

Una vez fijas las luces a utilizar (en este caso 1,2 y 3) para despejar la normal en cada píxel, debemos resolver el sistema en cada píxel. Es decir, estaremos traingulando una y otra vez una matriz dónde lo único que cambia es el término a la derecha de la igualdad. Por lo tanto, es interesante plantearse si existe una forma de evitar aplicar Gauss en cada punto. \\

Por lo explicado anteriormente (independencia lineal de las luces) sabemos que podemos aplicar Gauss normalmente sin encontrarnos con ceros en la diagonal. Es decir, podemos terminar el proceso de triangulación sin tener que haccer ninguna permutación de filas. Por lo visto en la clase teórica, esta condición es suficiente para hallar una factorización LU. \\

Dada una matriz $A$, la factorización LU consiste en encontrar dos matrices: una matriz $L$ triangular inferior con unos en la diagonal y una matriz $U$ triangular superior, de forma que se cumpla

\begin{center}
$A$ = $L$.$U$
\end{center}


 (con el producto usual de matrices)




\todo[inline]{Que es esto}
\todo[inline]{Por que nos sirve para nuestro problema, hablar de repeticion de calculos}


\section{Estimacion de profundidades}
\todo[inline]{Hablar del ultimo sistema de ecuaciones de la cual (aun) no tengo idea}

\subsection{Cholesky}
\todo[inline]{algo}

\section{Resultados}
\todo[inline] {Dar imagenes minimo del resultado final}


\section{Experimentación}

\todo[inline]{PENSAR MAS}

\todo[inline]{Mostrar como con diferentes luces obtenemos diferentes normales}

\todo[inline]{Diferencias de tiempos EG vs LU, con si y mascara mascara}

\todo[inline]{Mostrar resultados finales con mismas luces pero propias vs catedra}

