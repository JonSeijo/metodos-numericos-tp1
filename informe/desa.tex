% Cita textual seccion Desarrolo:

% Deben explicarse los metodos numericos que utilizaron
% y su aplicacion al problema concreto involucrado en el trabajo practico.
% Se deben mencionar los pasos que siguieron para implementar
% los algoritmos, las dificultades que fueron encontrando y la
% descripcion de como las fueron resolviendo.

% Explicar tambien como fueron planteadas y realizadas las
% mediciones experimentales. Los ensayos fallidos, hipotesis y
% conjeturas equivocadas, experimentos y metodos malogrados deben
% figurar en esta seccion,
% con una breve explicacion de los motivos de estas fallas
% (en caso de ser conocidas).

\section{Calibración}

\todo[inline]{Que es y por que tuvimos que hacerlo. Contar como necesitamos esto para aplicarlo al resto del trabajo practico}
\todo[inline]{Explicar como obtenemos ciertos datos (centro y punto mas iluminado)}
\todo[inline]{Explicar la idea teorica de la calibracion, incluir grafico esfera para ilustrar}
\todo[inline]{Escribir lindas las cuentas de los despejes}
\todo[inline]{En la seccion de experimentacion se vera como diffieren con las dadas por la catedra}


\section{Métodos}

\subsection{Eliminación Gaussiana}

\todo[inline]{Hablar del algoritmo}
\todo[inline]{Explicar por que podemos aplicarlo con las luces, decir que las 3 que tomamos son li entonces no pincha nunca}

\subsection{LU}
\todo[inline]{Que es esto}
\todo[inline]{Por que nos sirve para nuestro problema, hablar de repeticion de calculos}

\subsection{Cholesky}
\todo[inline]{algo}


\section{Cálculo de normales}

\todo[inline]{Explicar que es lo que tenemos que resolver}
\todo[inline]{Explicar como aplicamos los metodos listados arriba para resolver el problema, y por que podemos hacerlo}


\section{Estimacion de profundidades}

\todo[inline]{Hablar del ultimo sistema de ecuaciones de la cual (aun) no tengo idea}


\section{Resultados}
\todo[inline] {Dar imagenes minimo del resultado final}



\section{Experimentación}

\todo[inline]{PENSAR MAS}
\todo[inline]{Mostrar como con diferentes luces obtenemos diferentes normales}
\todo[inline]{Diferencias de tiempos EG vs LU vs mascara y combinaciones}

