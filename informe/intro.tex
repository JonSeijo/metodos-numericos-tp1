\section{Introducción}

% Pautas de tp hablan de "breve introduccion"

Este trabajo consiste en la reconstrucción de objetos 3D basándose en imágenes producidas con cámaras tradicionales, utilizando la técnica de \textit{fotometría estéreo}. Mostraremos que utilizando luces provenientes de diferentes ángulos, podemos aproximar las normales a la superficie y estimar las profundidades de cada punto. \\

En primer lugar resolveremos el problema de la calibración, es decir, encontraremos a partir de imágenes de una esfera cuáles son los diferentes vectores de iluminación. Utilizaremos estos datos para calcular las normales a la superficie en cada punto. Para esto nos enfrentaremos con varios sistemas de ecuaciones lineales, los cuales resolveremos algorítmicamente de forma matricial. Usaremos el método clásico de Eliminación Gaussiana, y nos aprovecharemos de la estructura de nuestro problema para encontrar una factorización LU e intentar reducir los tiempos de cómputo. \\

En segunda instancia, para el cálculo de profundidades aprovecharemos la forma de la matriz final para crear una nueva estructura más eficiente, donde aplicaremos el algoritmo de Cholesky para resolver el sistema. \\

En último lugar, realizaremos experimentos que intentarán mostrar cómo se comporta nuestro sistema tanto en la calibración, como en la obtención de normales y los cálculos de profundidad. Veremos cuáles son los resultados finales y cómo las diferentes elecciones de luces repercuten en el modelo generado. \\