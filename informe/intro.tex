\section{Introducción}

% Pautas de tp hablan de "brave introduccion"

Este trabajo consiste en la digitalización de objetos 3D basándose en imágenes producidas con cámaras tradicionales, utilizando la técnica de \textit{fotometría estéreo}. Mostraremos que utilizando luces provenientes de diferentes ángulos, podemos aproximar las normales a la superficie y estimar las profundidades de cada punto.

Para esto debemos resolver varios sistemas de ecuaciones lineales, los cuáles resolveremos algorítmicamente de forma matricial. Usaremos en un primer caso el método clásico de Eliminación Gaussiana, y veremos como utilizando factorización LU podemos reducir los tiempos de cómputo. Luego, utilizaremos la factorización de Cholesky.

Los experimentos .... []
\todo[inline]{Completar breve pantallazo a experimentos}